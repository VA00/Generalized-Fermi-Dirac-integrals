\documentclass{article}
\usepackage[intlimits]{amsmath}
%\usepackage{bigints}

\begin{document}

Library \texttt{libfermidirac} provides function:

\begin{equation}
\label{Ffermi}
F_k(\eta, \theta) 
= 
\int_0^\infty
\frac{x^k \; \sqrt{1+ \frac{\theta x}{2}}}{1+e^{x-\eta}} \; dx.
\end{equation}


Call it from C with
\begin{flushleft}
\tt
double Ffermi(double k, double eta, double theta);
\end{flushleft}


From definition \eqref{Ffermi} $k>-1$ and $\theta \geq 0$. For out-of-range parameters
function should return NaN.

The Fermi-Dirac integrals $G_n^{\pm}$ are:
\begin{equation}
\label{Gfermi}
G_n^{\pm}(\alpha, \beta) =
\frac{1}{\alpha^{3+2 n}} \int_{\alpha}^{\infty} \frac{x^{2 n+1 } \sqrt{x^2-\alpha^2}}{1+\exp(x\pm \beta)} \; dx.
\end{equation}
Usually: $\alpha = m_e/kT$, $\beta = \mu_e/kT$, $x=E/kT$.

To call from C use:\\
for function $G_n^+(\alpha,\beta)$:
\begin{flushleft}
\tt
double Gp(double n, double alpha, double beta);
\end{flushleft}


for function $G_n^-(\alpha,\beta)$:
\begin{flushleft}
\tt
double Gm(double n, double alpha, double beta);
\end{flushleft}





\end{document}
